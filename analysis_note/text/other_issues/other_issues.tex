\chapter{Other issues}
\label{Sect:issues}


\section{The cross section dependence on the beam energy}


The phi-integrated virtual photoproduction cross section $\sigma_{v}$ can be decomposed into the combination of the structure functions,\footnote[1]{The full decomposition (for the case of the unpolarized electron beam) can be found e.g. in Refs.~\cite{twopeg,Skorodumina:2016pnb} }
\begin{equation}
\sigma_{v} = \sigma_{T} + \varepsilon_{L}\sigma_{L}\text{~~~with~~~}\varepsilon_{L}=\frac{Q^{2}}{\nu^{2}}\varepsilon_{T},\label{eq:beam_en_dep}
\end{equation}
where $\sigma_{T}$ and $\sigma_{L}$ are the transverse and longitudinal structure functions, respectively, while $\varepsilon_{L}$ is the longitudinal polarization of the virtual photon with $\varepsilon_{T}$ given by Eq.~\eqref{polarization}.

Being decomposed in this way, the cross section $\sigma_{v}$ has a specific beam energy dependence, which is incorporated into the coefficient $\varepsilon_{L}$. The structure functions themselves, meanwhile, do not depend on the beam energy. A single experiment conducted with a certain beam energy allows for the extraction of $\sigma_{v}$ as a whole without accessing the separate structure functions. Thus, the beam energy dependence turns out to be implicitly incorporated into the extracted cross sections.

Although the experiment is conducted with a fixed value of the laboratory beam energy, the actual energy of the incoming electron involved in the reaction turns out to alter and differ from the fixed laboratory value due to  (i) the radiative effects that electrons undergo and (ii) the Fermi motion of the target proton. As a consequence, the extracted cross section cannot be associated with a distinct value of the electron beam energy, and this may complicate the interpretation of the results. Let's address these issues in more detail. 

\begin{itemize}


\item [(i)] The incoming and scattered electrons can emit photons thus reducing their energy. Due to the change of the incoming electron energy, the extracted cross sections correspond to the superposition of various beam energies. The correction due to this effect is included into the procedure of radiative corrections (see Sect.~\ref{Sect:rad_corr}).


\item [(ii)] The experiment off the moving proton with fixed laboratory beam energy corresponds to that off the proton at rest performed with varying effective beam energies~\cite{twopeg-d}. As a result, the extracted cross sections off moving protons are convoluted with the dependence of the quantity  $\varepsilon_{L}$ on the beam energy (see Eq.~\eqref{eq:beam_en_dep}). A study in Ref.~\cite{twopeg-d}, however, proves that this effect has an insignificant influence on the cross section. The correction due to this effect (which is negligible anyway) is automatically included into the procedure of unfolding the effects of the target motion (see Sect.~\ref{Sect:fermi_corr})\footnote[2]{Note that the radiative effects decrease the beam energy, while the Fermi motion leads to a symmetrical spread of the effective beam energy around the laboratory value.}. 


\end{itemize}

% The common way of managing with this problem is applying the radiative corrections, which are described in Sect.~\ref{Sect:rad_corr}. Once corrected, the cross section is attributed to the distinct value of the laboratory beam energy.


Being corrected, the cross sections extracted in this analysis may be assigned to the distinct value of the laboratory beam energy of $E_{beam} = 2.039$~GeV.
%However, in the experiment the information on these emissions is lost, and one has to assume the electron energy to be unchanged, thus introducing a systematic distortion to the extracted cross sections.

%Electron scattering off the moving proton performed with the beam energy $E_{beam}$ corresponds to that off the proton at rest conducted with the effective beam energy $\widetilde{E}_{beam}$, which is determined by the boost from the Lab frame to the quasi-Lab system, where the proton is at rest, while the incoming electron moves along the $z$-axis~\cite{twopeg-d}. This effective beam energy thus depends on the Fermi momentum of the target proton and differs event by event. Therefore, the experiment off the moving proton with the fixed electron beam energy corresponds to that off the proton at rest performed with the altered beam energy.


%The virtual photoproduction cross sections $\sigma_{v}$ has a specific beam energy dependence, which originates from two sources. The first source is the explicit beam energy dependence of the quantities $\Gamma_{v}$ and $\varepsilon_{T}$, which are given by Eq.~\eqref{flux} and Eq.~\eqref{polarization}, respectively. In the proton at rest experiments the conventional practice is to determine these two quantities in the Lab frame, where the electron beam has the fixed energy. Generally speaking, in the experiments off moving protons these quantities should be determined in the quasi-Lab system, and altered event by event value of the effective beam energy should be used in the calculations. Beside that, the $\sigma_{v}$ has also an implicit dependence on the beam energy that is incorporated into the coefficient $\varepsilon_{L}$ of the cross section decomposition into the transverse and longitudinal parts ($\sigma = \sigma_{T}+\varepsilon_{L}\sigma_{L}$). Therefore, in the moving proton experiments the $\sigma_{v}$ turns out to be convoluted with the dependencies of the quantities $\varepsilon_{T}$, $\varepsilon_{L}$, and $\Gamma_{v}$ on the beam energy. As a consequence, the extracted cross section cannot be associated with the distinct value of the electron beam energy that may complicate the interpretation of the result and its comparison with the cross sections of the proton at rest experiment.

%This question is addressed in detail in Ref.~\cite{twopeg-d}. This study proves that the convolution of the cross section with the dependencies of the quantities $\varepsilon_{T}$, $\varepsilon_{L}$, and $\Gamma_{v}$ on the beam energy has an insignificant influence on it.


%In this analysis the cross section is calculated ignoring its beam-energy dependence, i.e. the quantities $\varepsilon_{T}$ and $\Gamma_{v}$ were calculated in the Lab frame using the fixed laboratory value of the beam energy $E_{beam}$ = 2.039 GeV. The correction due to this approximation (which is negligible anyway) is automatically included into the procedure of unfolding the effects of the target motion (see Sect.~\ref{Sect:fermi_corr}). Being corrected, the extracted cross sections may be assigned to a distinct value of the laboratory beam energy.



\section{Off-shell effects}

The target proton is bound in the deuterium nucleus and thus undergoes nucleon-nucleon interactions. The nucleon mass, however, is thought to be an interaction-dependent quantity, i.e. the nucleon's physical mass in a nucleus is smaller than that of a free nucleon~\cite{Noble:1980my}. In other words, the target proton bound in the deuteron is off-shell, which means that its four-momentum squared is not equal to its mass squared.

In the study~\cite{Ye_Tian:2017}, which aimed at $\pi^{-}$ electroproduction off the neutron in deuterium, the impact of the off-shell effects on the measured cross sections was shown to be marginal. In this study the off-shell effects are ignored.














