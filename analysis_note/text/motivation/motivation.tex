\chapter{Introduction}
\label{Sect:motiv}


Exclusive meson photo- and electroproduction reactions off protons are intensively studied in laboratories all over the world as a very powerful tool for the investigation of nucleon structure and the principles of the strong interaction. These studies include the extraction of various observables through the analysis of the experimental data and the consequent theoretical and phenomenological interpretations of these observables~\cite{Krusche:2003ik,Aznauryan:2011qj,Skorodumina:2016pnb}.


%During the last decades great efforts have been performed in laboratories all over the world in order to investigate exclusive reactions of meson photo- and electroproduction off protons as they are a very powerful tool to the understanding of nucleon structure and principles of the strong interaction. This investigation is typically carried out through the detailed analysis of the experimental data with the goal of extracting various observables. Further theoretical and phenomenological interpretations of the extracted observables provide the desired valuable information on nucleon structure and features of the strong interaction [1?4].

By now exclusive reactions off the free proton have been studied in considerable detail, and a lot of information on differential cross sections and different single and double-polarization asymmetries with almost complete coverage of the final hadron phase-space is available. A large part of this information came from the analysis of data collected in Hall B at Jefferson Lab with the CLAS detector~\cite{Mecking:2003zu} and stored in the CLAS physics database~\cite{CLAS_DB}.


%\footnote[1]{The inclusive structure function $F_{2}$ measured off the deuteron also demonstrates }.

Meanwhile, reactions occurring in photon and electron scattering off nuclei are less extensively investigated, i.e. the experimental information on these processes is scarce and mostly limited to the inclusive measurements of total nuclear photoproduction cross sections~\cite{Mokeev:1995fy,Bianchi:1994ax,Ahrens:1986hn} and nucleon structure function $F_{2}$~\cite{Osipenko_2005_note,Osipenko:2005gt,Osipenko:2010sb}. The available inclusive data, however, exhibit some surprising peculiar features not fully elucidated over the years, which are now attracting significant scientific attention. Specifically, the nuclear photoproduction cross section (per nucleon) turns out to be less pronounced and damped in strength compared with the cross section off the free proton. This effect manifests itself differently depending on the invariant mass range, i.e. the $\Delta(1232)$-resonance peak is damped, but still evident for all nuclei, however, the second resonance region becomes somewhat less pronounced and damped for the deuteron and strongly suppressed and structureless for all heavier nuclei. A similar effect is observed in the behavior of the nucleon structure function $F_{2}$, which in the case of the deuteron shows moderate damping and flattening~\cite{Osipenko:2005gt} and completely loses its structure, when measured off carbon~\cite{Osipenko:2010sb} (compared with the free proton structure function~\cite{Osipenko:2003bu}). A fact of particular interest is that the intensity of this effect increases as $Q^{2}$ grows, i.e. as $Q^{2}$ = 3~GeV$^{2}$ is reached, the structure function $F_{2}$ becomes almost flat even for the deuteron~\cite{Osipenko:2010sb}. These peculiar features can not be explained by the Fermi motion of nucleons in the nucleus and are thought to be an indication that nucleons and their exited states, bound inside the nuclear medium, may be subject to some modifications of their properties~\cite{Mokeev:1995fy,Bianchi:1994ax,Ahrens:1986hn,Krusche:2004xz,Noble:1980my}. 



%The inclusive structure function $F_{2}$ measured off the deuteron also demonstrates damping and smoothing of the structure comparing with that measured off the free proton, and the intensity of this effect increases as the $Q^{2}$ grows.



%Similar effect is seen in the behavior of the inclusive structure function $F_{2}$ measured off the deuteron, which demonstrate damping and smoothing of the structure comparing with the proton measurement.  

%is scarce and mostly limited to the data on total nuclear photoproduction cross sections


This phenomenon, which is still not fully understood, generates lots of debates among scientists, triggering efforts to describe the processes that happen in reactions off bound nucleons. These studies rely heavily on the experimental data, which at the moment are mostly limited to inclusive measurements~\cite{Mokeev:1995fy,Bianchi:1994ax,Ahrens:1986hn,Osipenko_2005_note,Osipenko:2005gt,Osipenko:2010sb} and lack information on exclusive reactions. This information, however, is crucial, since various exclusive channels have different energy dependencies and different sensitivity to reaction mechanisms. This situation creates a strong demand for exclusive measurements off bound nucleons, and the deuteron, being the lightest and weakly-bound nucleus, is the best target for initiating these efforts.
%to begin

This study provides the first results of cross section measurements for the exclusive process of charged double-pion electroproduction off the proton bound in the deuteron. The results are obtained through the analysis of experimental data on electron scattering off the deuteron target, collected with the CLAS detector. The measurements are performed in the second resonance region, where the double-pion production plays an important role, i.e. the channel opens at the double-pion production threshold at $W \approx 1.22$~GeV, contributes significantly to the total inclusive cross section for $W \lesssim 1.6$~GeV, and starts to dominate all other exclusive channels for $W \gtrsim 1.6$~GeV .


The experimental identification of exclusive multi-particle final states is a rather sophisticated task, which requires certain analysis techniques to be elaborated and established. This was carried out over the last twenty years as the different studies of double-pion production off the free proton were being performed~\cite{Rip_an_note:2002,Ripani:2002ss,Fed_an_note:2007,Fedotov:2008aa,Isupov:2017lnd,Golovach,Arjun,Fed_an_note:2017,Fed_paper_2018}, and currently a solid framework for such studies is in place. For this particular study, focused on the $N\pi\pi$ final state, this framework laid the foundation. However, the deuteron as a target introduces some specific issues, which are external to the free proton data analysis and originate from (a) the motion of the target proton in the deuteron and (b) complex effects of the final state interactions due to the presence of the additional nucleon. This caused some difficulties that were encountered and needed to be overcome during the analysis and, therefore, in this report special attention is paid to a detailed description of these issues.

Specifically, the report presents the integrated and single-differential cross sections of the reaction $\gamma_{v}p(n) \rightarrow p' (n')\pi^{+}\pi^{-}$ in the kinematic region of invariant mass $W$ from 1.3~GeV to 1.825~GeV and photon virtuality $Q^{2}$ from 0.4~GeV$^2$ to 1~GeV$^2$. Sufficient experimental statistics allows narrow binning, e.g. 25~MeV in $W$ and 0.05~GeV$^2$ in $Q^2$, while maintaining an adequate statistical uncertainty. Cross sections are extracted in the quasi-free regime, which implies that only events not affected by final state interactions were selected.

This study benefits from the fact that the corresponding cross sections of the same exclusive reaction off the free proton have been recently extracted from CLAS data~\cite{Fed_an_note:2017,Fed_paper_2018}. These free proton measurements were performed under the same experimental conditions as the cross sections of this study, including the beam energy value and the target setup. For the majority of $(W,~Q^{2})$ points, the statistical uncertainty combined with the model dependent uncertainty ($\delta_{\text{stat,~mod}}^{\text{tot}}$) is on a level of $\sim$ 1\%-3\% for the free proton integral cross sections and on a level of $\sim$ 4\%-6\% for the quasi-free integral cross sections obtained in this study. Being performed in the same experimental configuration, both measurements have identical binning in all kinematic variables and similar inherent systematic inaccuracies. Therefore, the direct comparison of these two sets of cross sections provides experimentally the best possible opportunity to investigate the differences and alterations (including possible in-medium modifications) that occur in the exclusive reaction off the bound proton in comparison with that off the free proton. This comparison also allows us to better understand the influence of Fermi motion and final state interactions on the cross sections.
%. Both measurements have, therefore, exactly the same binning in all kinematic variables and similar inherent systematic inaccuracies.
% (which included both hydrogen and deuterium target runs in the same experimental configuration) and have therefore exactly the same binning in all kinematic variables as the cross sections of this study.
%cross sections were obtained under the same experimental conditions (including the beam energy value and the target setup) as the cross sections of this study and have exactly the same binning in all kinematic variables.



A few example plots, which demonstrate the difference between integral cross sections obtained in this analysis and their free proton analogue from Ref.~\cite{Fed_an_note:2017,Fed_paper_2018} are given in Sect.~\ref{Sect:concl} of this report. Meanwhile, the complete compilation of this comparison as well as the full physical discussion of the results and their physical interpretation will be presented in the PhD thesis (which is in preparation) and a future publication on the subject.





